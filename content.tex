
\section{Innledning}
Lorem ipsum dolor sit amet, consectetur adipiscing elit. Mauris leo nunc, tincidunt elementum sem id, varius sodales dolor. Fusce sit.
Mauris a mauris tortor. Nulla congue quam a congue feugiat. Vestibulum ligula lectus, rutrum eget vulputate a, vehicula nec leo.

\section{Foobar}
\begin{enumerate}
\item Item
\item One more intem
\end{enumerate}

Aliquam erat volutpat. Pellentesque luctus neque felis. In egestas efficitur mi, semper vestibulum velit elementum in. Donec ut urna quam.


\section{Inkrementell utvikling}
Nunc at urna nec nibh commodo blandit. Aliquam in nibh maximus, congue enim quis, lobortis erat. Suspendisse vitae leo dui. Donec rutrum id quam vitae laoreet. Sed felis sem, cursus vel lacus at, ornare cursus tellus. Phasellus pulvinar turpis et hendrerit congue. Nulla elementum mollis tellus vel finibus. Sed eget volutpat felis. Praesent ac sapien quis sapien facilisis fringilla. Cras laoreet, lectus ac dignissim faucibus, odio turpis vehicula arcu, et venenatis lorem turpis aliquet tortor. Phasellus nulla arcu, imperdiet id ultrices sit amet, iaculis vitae justo.


Den inkrementelle utviklingsmodellen er en evolusjonær model \cite{wiki:incremental_build_model}. Man starter med en klar modell for det endelige systemet, 
Disse to metodene vil følge hverandre der inkrementell utvikling gjennomføres med en iterativ prosess.
Dette omtales gjerne også sammen som \textit{iterative and incremental development} eller \textit{IID}.


\section{Utviklingsmetoder}
Alle de evolusjonære utviklingsprosessene er basert på inkrementell utvikling \cite{larman2003iterative}. Evolusjonen i prosessen 
\textcite{royce1970software} beskriver imidlertid i sin opprinnelige artikkel der han beskriver \textit{waterfall} at man kan 
gjennomføre utviklingen i 2 runder, som også kan sees på som inkrementer.

\subsection{Fordeler og ulemper}
flight-control-system eller en røntgen-maskin kan ikke brukes hvis ikke hele systemet er operativt. Samtidig viser
\textcite{larman2003iterative} til at NASA brukte en IID-basert utviklingsmetode for avionics-programvaren til den første romfergen.

